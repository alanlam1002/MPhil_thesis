%!TEX root = ../thesis.tex
% ******************************* Thesis Appendix A ****************************
\chapter{Useful relations for implementation of constrained scheme}
\label{A1}

\section{The elliptic equations in constrained scheme}
We use orthonormal-basis for the vector fields and tensor fields in the elliptic equations in section \ref{section_XCFC} and \ref{section1.5.3}.
The expression and discretization of the equations in cylindrical and spherical coordinate is non-trivial under orthonormal-basis.
We list the relations in various geometry in the following.

\subsection{Cylindrical coordinate}
The scalar Laplacian in cylindrical coordinate is given by
\begin{align}
    \mathcal{D}^2 u &= \frac{1}{r}\frac{\partial }{\partial r} \left( r \frac{\partial u}{\partial r}\right)
            + \frac{1}{r^2} \frac{\partial^2 u}{\partial \phi^2} 
            + \frac{\partial^2 u}{\partial z^2},
\end{align}
while the corresponding discretization is given by
\begin{align}
\begin{split}
    \mathcal{D}^2 u_{i,j,k} &= \frac{ r_{i+1/2} \left(u_{i+1,j,k} - u_{i,j,k} \right)
            - r_{i-1/2} \left(u_{i,j,k} - u_{i-1,j,k} \right)}{r_i \Delta r^2} \\
            &+ \frac{ u_{i,j+1,k} - 2 u_{i,j,k} + u_{i,j-1,k}}{r_i^2 \Delta \phi^2}
            + \frac{ u_{i,j,k+1} - 2 u_{i,j,k} + u_{i,j,k-1}}{\Delta z^2}.
\end{split}
\end{align}
For a generic vector $\mathbf{X} = \left(X^r, X^\phi, X^z \right)$, its divergence is given by
\begin{align}
    \mathcal{D}_i X^i &= \frac{1}{r}\frac{\partial \left(r X^r\right)}{\partial r} + \frac{1}{r}\frac{\partial X^\phi}{\partial \phi}
            + \frac{\partial X^z}{\partial z},
\end{align}
and the vector Laplacian is
\begin{align}
    \left(\Delta_L \mathbf{X}\right)^r &= \mathcal{D}^2 X^r - \frac{2}{r^2}\frac{\partial X^\phi}{\partial \phi}
            + \frac{1}{3}\frac{\partial}{\partial r}\left( \mathcal{D}_i X^i \right), \\
    \left(\Delta_L \mathbf{X}\right)^\phi &= \mathcal{D}^2 X^\phi - \frac{X^\phi}{r^2} + \frac{2}{r^2}\frac{\partial X^r}{\partial \phi}
            + \frac{1}{3 r}\frac{\partial}{\partial \phi}\left( \mathcal{D}_i X^i \right), \\
    \left(\Delta_L \mathbf{X}\right)^z &= \mathcal{D}^2 X^z + \frac{1}{3}\frac{\partial}{\partial z}\left( \mathcal{D}_i X^i \right).
\end{align}
The generalized Dirac gauge conditions in cylindrical coordinate is given by
\begin{align}
    \left(\mathcal{D}_j h^{ij}\right)^r &= \mathcal{D}_j h^{rj} - \frac{h^{\phi\phi}}{r}, \\
    \left(\mathcal{D}_j h^{ij}\right)^\phi &= \mathcal{D}_j h^{\phi j} + \frac{h^{r\phi}}{r}, \\
    \left(\mathcal{D}_j h^{ij}\right)^z &= \mathcal{D}_j h^{z j}.
\end{align}

\subsection{Spherical coordinate}
The scalar Laplacian in spherical coordinate is given by
\begin{align}
    \mathcal{D}^2 u &= \frac{1}{r^2} \frac{\partial}{\partial r} \left( r^2 \frac{\partial u}{\partial r} \right)
        + \frac{1}{r^2 \sin{\theta}} \frac{\partial}{\partial \theta} \left( \sin{\theta} \frac{\partial u}{\partial \theta} \right) 
        + \frac{1}{r^2 \sin^2{\theta}} \frac{\partial^2 u}{\partial \phi^2}
\end{align}
while the corresponding discretization is given by
\begin{align}
\begin{split}
    \mathcal{D}^2 u_{i,j,k} =& \frac{r_{i+1/2}^2 \left(u_{i+1,j,k} - u_{i,j,k} \right)
            - r_{i-1/2}^2 \left(u_{i,j,k} - u_{i-1,j,k} \right)}{r_i^2 \Delta r^2} \\
            &+ \frac{ \sin{\theta_{j+1/2}} \left(u_{i,j+1,k} - u_{i,j,k} \right)
            - \sin{\theta_{j-1/2}} \left(u_{i,j,k} - u_{i,j-1,k} \right)}{r_i^2 \sin{\theta_j} \Delta \theta^2} \\
            &+ \frac{ u_{i,j,k+1} - 2 u_{i,j,k} + u_{i,j,k-1}}{r_i^2 \sin^2{\theta_j}\Delta \phi^2}.
\end{split}
\end{align}
For a generic vector $\mathbf{X} = \left(X^r, X^\theta, X^\phi \right)$, its divergence is given by
\begin{align}
    \mathcal{D}_i X^i = \frac{1}{r^2} \frac{\partial \left(r^2 X^r\right)}{\partial r} 
            + \frac{1}{r\sin\theta}\frac{\partial \left(\sin\theta X^\theta\right)}{\partial \theta}
            + \frac{1}{r\sin\theta}\frac{\partial X^\phi}{\partial \phi}
\end{align}
and the vector Laplacian is
\begin{align}
    \left(\Delta_L \mathbf{X}\right)^r &= \mathcal{D}^2 X^r 
            - \frac{2}{r^2}\left[ X^r + \frac{1}{\sin\theta}\frac{\partial \left(\sin\theta X^\theta\right)}{\partial \theta} 
            + \frac{1}{\sin\theta} \frac{\partial X^\phi}{\partial \phi}\right]
            + \frac{1}{3}\frac{\partial}{\partial r}\left( \mathcal{D}_i X^i \right), \\
    \left(\Delta_L \mathbf{X}\right)^\theta &= \mathcal{D}^2 X^\theta 
            + \frac{2}{r^2}\frac{\partial X^r}{\partial \theta} - \frac{X^\theta}{r^2 \sin^2\theta}
            - \frac{2 \cos\theta}{r^2\sin^2\theta}\frac{\partial X^\phi}{\partial \phi}
            + \frac{1}{3 r}\frac{\partial}{\partial \theta}\left( \mathcal{D}_i X^i \right), \\
    \left(\Delta_L \mathbf{X}\right)^\phi &= \mathcal{D}^2 X^\phi 
            - \frac{X^\phi}{r^2 \sin^2\theta} + \frac{2}{r^2\sin\theta}\frac{\partial X^r}{\partial \phi}
            + \frac{2 \cos\theta}{r^2\sin^2\theta}\frac{\partial X^\theta}{\partial \phi}
            + \frac{1}{3 r\sin\theta}\frac{\partial}{\partial \phi}\left( \mathcal{D}_i X^i \right).
\end{align}
The generalized Dirac gauge conditions in spherical coordinate is given by
\begin{align}
    \left(\mathcal{D}_j h^{ij}\right)^r &= \mathcal{D}_j h^{rj} - \frac{h^{\theta\theta}+h^{\phi\phi}}{r}, \\
    \left(\mathcal{D}_j h^{ij}\right)^\theta &= \mathcal{D}_j h^{\theta j} + \frac{h^{r\theta} - \cot\theta h^{\phi\phi}}{r}, \\
    \left(\mathcal{D}_j h^{ij}\right)^\phi &= \mathcal{D}_j h^{\phi j} + \frac{h^{r\phi} + \cot\theta h^{\theta\phi}}{r}.
\end{align}
