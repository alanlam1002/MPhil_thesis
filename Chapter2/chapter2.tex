%!TEX root = ../thesis.tex
%*******************************************************************************
%*********************************** Second Chapter *****************************
%*******************************************************************************

\chapter{Formulations of the Relativistic Hydrodynamics}  %Title of the Second Chapter

\ifpdf
    \graphicspath{{Chapter2/Figs/PDF/}{Chapter2/Figs/}}
\else
    \graphicspath{{Chapter2/Figs/}}
\fi


%********************************** %First Section  **************************************
\section{Relativistic Hydrodynamics} %Section - 2.1
In the previous chapter, we explored the formulation of the left hand side of the Einstein equations.
In this chapter, we focus on the right hand side of the equations,
specifically the 3+1 treatment of perfect fluid in conservative form \cite{banyuls1997numerical,rezzolla2013relativistic}.

\subsection{Perfect fluid}
\label{section2.1.1}
In this section, we consider relativistic \textit{perfect fluid} (i.e. no viscous effects and heat fluxes).
The energy-momentum tensor of perfect fluid is given by
\begin{align}
    T^{\mu\nu} = \rho h u^\mu u^\nu + p g^{\mu\nu},
\end{align}
where $\rho$ is the rest-mass density, 
$u^\mu$ is the fluid 4-velocity,
$h = 1 + \epsilon + \frac{p}{\rho}$ is the enthalpy, 
$p$ is the pressure, 
and $\epsilon$ is the specific internal energy.\\
The \textit{Lorentz factor} $W$ of the fluid with respects to the Eulerian observers is given by 
the projection of fluid 4-velocity $u^\mu$ to $n^\mu$
\begin{align}
    W \coloneqq - n_{\mu} u^\mu = \alpha u^t.
\end{align}
Thus, the spatial 4-velocity of a fluid measured by Eulerian observers is
\begin{align}
    v^\mu \coloneqq \frac{\gamma_{\nu}{}^{\mu} u^\nu} {W},
\end{align}
or in component form
\begin{align}
    v^t &= 0, & v^i &=  \frac{u^i}{W} + \frac{\beta^i}{\alpha}, \\
    v_t &= \beta_i v^i, & v_i &= \frac{u_i}{W}.
\end{align}
As a result, the \textit{Lorentz factor} can also be written as
\begin{align}
    W = \frac{1}{\sqrt{1-v_i v^i}}.
\end{align}
The dynamics of fluid are governed by the equations of rest-mass conservation and energy momentum conservation.
The system is closed by the equation of state (EOS) of the fluid,
which reflects the thermodynamic properties of the fluid at a macroscopic level.
Here, we consider two common analytic equation of state as follows:
\begin{enumerate}
    \item Ideal-gas
    \begin{align}
        p = (\Gamma-1)\rho \epsilon,
    \end{align}
    where $\Gamma$ is the polytropic index of the gas.
    \item Polytropic EOS 
    \begin{align}
        p = K\rho^\Gamma,
    \end{align}
    where $\Gamma$ again is the polytropic index the gas, and $K$ is a constant of proportionality.
\end{enumerate}

\subsection{Rest-mass conservation}
\label{section2.1.2}
The general-relativistic conservation of rest-mass is given by
\begin{align}
    \nabla_\mu J^\mu &= 0, & J^\mu &\coloneqq \rho u^\mu,
\end{align}
where $J^\mu$ is the rest-mass current density.
Under 3+1 decomposition, the rest-mass conservation can be written as
\begin{align}\label{eq:mass_cons}
    \partial_t \left(\sqrt{\gamma}D\right) + \partial_i \left[\sqrt{\gamma} D \left(\alpha v^i - \beta^i \right) \right] = 0,
\end{align}
where
\begin{align}
    D \coloneqq \rho \alpha u^t = \rho W,
\end{align}
is the conserved quantity.

\subsection{Energy and momentum conservation}
\label{section2.1.3}
Other than the conservation of rest-mass, the energy and momentum conservation of the fluid must also be satisfied
\begin{align}
    \nabla_\nu T^{\mu\nu} = 0.
\end{align}
Following the 3+1 decomposition of stress-energy tensor introduced in equation (\ref{eq:T_decompose}),
the fluid quantities in the case of perfect fluid become
\begin{align}
    S^{\mu\nu} &= \rho h W^2 v^\mu v^\nu + p \gamma^{\mu\nu}, \\
    S^\mu &= \rho h W^2 v^\mu, \\
    E &= \rho h W^2 - p.
\end{align}
The conservation of energy and momentum hence leads to
\begin{align}
    &\partial_t \left( \sqrt{\gamma} S_j \right) + \partial_i \left[\sqrt{\gamma} \left(\alpha S^i{}_j - \beta^i S_j\right) \right]
    = \sqrt{\gamma} \left(\frac{1}{2} \alpha S^{ik} \partial_j \gamma_{ik} + S_i \partial_j \beta^i - E \partial_j \alpha\right), \label{eq:mom_cons} \\
    &\partial_t \left( \sqrt{\gamma} E \right) + \partial_i \left[ \sqrt{\gamma} \left(\alpha S^i - \beta^i E \right) \right]
    = \sqrt{\gamma} \left(\alpha K_{ij} S^{ij} - S^i \partial_i \alpha \right). \label{eq:E_cons}
\end{align}
Equation (\ref{eq:E_cons}) can also be written as
\begin{align}
    &\partial_t \left( \sqrt{\gamma} \tau \right) + \partial_i \left[\sqrt{\gamma} \left(\alpha\left(S^i - D v^i \right) - \beta^i \tau \right)\right]
    = \sqrt{\gamma} \left(\alpha K_{ij} S^{ij} - S^i \partial_i \alpha \right). \label{eq:tau_cons},
\end{align}
to recover the energy density in Newtonian limit.
where the conserved quantity $\tau$ is defined as
\begin{align}
    \tau \coloneqq E - D.
\end{align}

\subsection{Conservative form of hydrodynamics equations}
\label{section2.1.4}
Here, we summarize \cref{eq:mass_cons,eq:mom_cons,eq:tau_cons} and express in conservative form
\begin{align}
    \partial_t \left(\sqrt{\gamma} \mathbf{Q} \right) + \partial_i \left(\sqrt{\gamma} \mathbf{F}^i \right) = \sqrt{\gamma} \mathbf{S},
\end{align}
where the conserved variables $\mathbf{Q}$, flux terms $\mathbf{F}^i$ and source terms $\mathbf{S}$ are
\begin{align}
    & \mathbf{Q} =
    \begin{pmatrix}
    D \\
    S_j \\
    \tau
    \end{pmatrix}
    =
    \begin{pmatrix}
    \rho W \\ 
    \rho h W^2 v_j \\ 
    \rho h W^2 - p - D
    \end{pmatrix},\\
    & \mathbf{F}^i =
    \begin{pmatrix}
    D\left(\alpha v^i -\beta^i \right) \\
    S_j \left( \alpha v^i - \beta^i \right) + \alpha \delta^i{}_j p \right) \\
    \tau left( \alpha v^i - \beta^i \right) + \alpha p v^i \right) 
    \end{pmatrix},\\
    & \mathbf{S} =
    \begin{pmatrix}
    0 \\
    \frac{1}{2} \alpha S^{ik} \partial_j \gamma_{ik} + S_i \partial_j \beta^i - E \partial_j \alpha \\
    \alpha K_{ij} S^{ij} - S^i \partial_i \alpha
    \end{pmatrix},
\end{align}
which is commonly known as the "\textit{Valencia formulation}" of the relativistic-hydrodynamics equatons. \\
The characteristic speeds of such system is given by
\begin{align}\label{eq:valencia}
    \lambda_0 &= \alpha v^i - \beta^i \qquad \text{(triple eigenvalue)},\\
    \lambda_{\pm} &= \frac{\alpha}{1-v^2 c_s^2} \left\{ v^i \left(1-c_s^2 \right) 
    \pm c_s^2 \sqrt{\left(1-v^2 \right) \left[\gamma^ii \left(1- v^2 c_s^2 \right) - {v^i}^2\left(1-c_s^2 \right) \right]} \right\},
\end{align}
where $\lambda_{\pm}$ measures the propagation speeds of the \textit{acoustic waves} of the system,
and $\lambda_0$ corresponds to the propagation speed of \textit{matter waves}.

\subsection{Conserved to Primitive variables conversion} 
\label{section2.1.5}
It is not trivial to convert the conserved variables $\left(D,S_i,\tau\right)$ introduced in section \ref{section2.1.4}
to primitive variables $\left(\rho, W, v^i, p\right)$ in relativistic hydrodynamics.
One must solve non-linear equations numerically.
In \texttt{Gmunu}, we adopted the routine used in \cite{galeazzi2013implementation}.
The implementation details are included here:
\begin{Step}
    \item Calculate the rescaled variables and some useful relations which are fixed during the iterations
    \begin{align}
        S &\coloneqq \sqrt{S_i S^i}, \\
        r &\coloneqq \frac{S}{D}, & q &\coloneqq \frac{\tau}{D}, & k &\coloneqq \frac{S}{\tau + D}.
    \end{align}
    \item Determine the bounds of the root
    \begin{align}
        z_- &\coloneqq \frac{k/2}{\sqrt{1-k^2/4}}, & z_+ &\coloneqq \frac{k}{\sqrt{1-k^2}}.
    \end{align}
    \item In the interval $\left[z_-, z_+ \right]$, we solve
    \begin{align}\label{eq:cons2prim_root}
        f(z) = z - \frac{r}{\hat{h}(z)},
    \end{align}
    where
    \begin{align}
        \hat{h}(z) &= \left(1+\hat{\epsilon}(z) \right) \left( 1+ \hat{a}(z) \right),
    \end{align}
    \begin{align}
        \hat{\epsilon}(z) &= \hat{W}(z)q - zr + \frac{z^2}{1+\hat{W}(z)},
        & \hat{a}(z) &= \frac{\hat{p}(z)}{\hat{\rho}(z)\left( 1 + \hat{\epsilon}(z) \right) },
    \end{align}
    \begin{align}
        \hat{p}(z) &= p\left(\hat{\rho}(z), \hat{\epsilon}(z) \right),
        & \hat{\rho}(z) &= \frac{D}{\hat{W}(z)},
        & \hat{W}(z) &= \sqrt{1+z^2}.
    \end{align}
    Equation (\ref{eq:cons2prim_root}) is solved using Illinois algorithm in \texttt{Gmunu}.
    Note that during the iterations, we enforce the density $\rho$ and the specific energy $\epsilon$ fail into the validity region of the EOS,
    i.e., we evaluate the updated $\rho$ and $\epsilon$ with 
    $\hat{\rho}(z) = \max\left(\min\left(\rho_\max{},\hat{\rho}\right),\rho_\min{}\right)$ and
    $\hat{\epsilon}(z) = \max\left(\min\left(\epsilon_\max{},\hat{\epsilon}\right),\epsilon_\min{}\right)$.
    \item With the root $z_0$ of \cref{eq:cons2prim_root},
    we can then work out the primitive variables $\left(\rho, \epsilon, p\right)$ respectively with the equations used in step 3.
    For the velocity $v^i$, it can be obtained with $z$ by
    \begin{align}
        \hat{v}^i(z) = \frac{S^i/D}{\hat{h}(z)\hat{W}(z)}.
    \end{align}
\end{Step}

%********************************** %Second Section  *************************************
\section{The reference-metric formalism} %Section - 2.2
\label{section2.2}
The reference-metric formalism, a generalization of the "\textit{Valencia}" formulation, was original proposed by Montero \cite{montero2014general} in 2014 for general relativistic hydrodynamics (GRHD)
and later extended to ideal magnetohydrodynamics.
With the conformal decomposition introduced in section \ref{section1.3.1},
the determinant of the spatial metric can be rewritten as
\begin{align}\label{eq:reference_metric_1}
    \begin{split}
    \sqrt{\gamma} &= \psi^6 \sqrt{\tilde{\gamma}} \\
    &= \sqrt{\hat{\gamma}} \psi^6 \sqrt{\tilde{\gamma}/\hat{\gamma}},
    \end{split}
\end{align}
where $\hat{\gamma}_{ij}$ is the \textit{time-independent reference metric}
which can be chosen to fit the mesh grid in curvilinear coordinate.
Therefore, the "\textit{Valencia}" formulation in equation (\ref{eq:valencia}) can be generalized as
\begin{align}
    \partial_t \mathbf{q} + \hat{\nabla}_i \mathbf{f}^i = \mathbf{s},
\end{align}
where $\hat{\nabla}_i$ is the corvariant derivatives associated with the time-independent reference metric $\hat{\gamma}_{ij}$,
$\left(\mathbf{q},\mathbf{f}^i,\mathbf{s}\right)$ are the $\textit{conformally rescaled}$ conserved variables,
fluxes and source terms respectively.\\
Comparing to the "\textit{Valencia}" formulation,
the hydrodynamical equations in reference-metric approach are numerically more accurate in curvilinear coordinates in multi-dimensional simulation,
especially when symmetry is imposed.
Equation (\ref{eq:reference_metric_1}) can be further written as
\begin{align}
    \partial_t \mathbf{q} + \frac{1}{\sqrt{\hat{\gamma}}}\partial_j \left(\sqrt{\hat{\gamma}} \mathbf{f}^j \right) = \mathbf{s} + \mathbf{s}_{geom},
\end{align}
where $\mathbf{s}_{geom}$ are the \textit{geometrical} source terms 
which contain the 3-Christoffel symbols $\hat{\Gamma}^i{}_{jk}$ associated with the reference metric $\hat{\gamma}_{ij}$. \\
Note that the momentum conservation in this expression satisfy to \textit{machine precision} rather than to the level of truncation error 
due to the fact that the geometrical source terms $\mathbf{s}_{geom}$ are identically vanishing 
for the components associated with negligible coordinates in the metric.

\subsection{General relativistic hydrodynamics (GRHD) equations}
\label{section2.2.1}
The equations of relativistic hydrodynamics in reference-metric approach is given by
\begin{align}
    \partial_t \left( q_{D} \right) + \frac{1}{\sqrt{\hat{\gamma}}} \partial_j \left[ \sqrt{\hat{\gamma}} \left( f_{D} \right)^j \right]
    &= 0, \\
    \partial_t \left( q_{S_i} \right) + \frac{1}{\sqrt{\hat{\gamma}}} \partial_j \left[ \sqrt{\hat{\gamma}} \left( f_{S_i} \right)^j \right]
    &= s_{S_i} + \hat{\Gamma}^l{}_{ik}\left( f_{S_l} \right)^k, \\
    \partial_t \left( q_{\tau} \right) + \frac{1}{\sqrt{\hat{\gamma}}} \partial_j \left[ \sqrt{\hat{\gamma}} \left( f_{\tau} \right)^j \right]
    &= s_{\tau}, \\
\end{align}
The conformally rescaled conserved quantities $\mathbf{q}$ are
\begin{align}
    q_D & \coloneqq \psi^6 \sqrt{\tilde{\gamma}/\hat{\gamma}} D = \psi^6 \sqrt{\tilde{\gamma}/\hat{\gamma}} \left( \rho W \right), \\
    q_{S_i} & \coloneqq \psi^6 \sqrt{\tilde{\gamma}/\hat{\gamma}} S_i = \psi^6 \sqrt{\tilde{\gamma}/\hat{\gamma}} \left( \rho h W^2 v_i \right), \\
    q_\tau & \coloneqq \psi^6 \sqrt{\tilde{\gamma}/\hat{\gamma}} \tau = \psi^6 \sqrt{\tilde{\gamma}/\hat{\gamma}} \left( \rho h W^2 - p - D \right).
\end{align}
The flux terms $\mathbf{f}^i$ are
\begin{align}
    \left(f_D \right)^i & \coloneqq \psi^6 \sqrt{\tilde{\gamma}/\hat{\gamma}} \left( D \hat{v}^i \right), \\
    \left(f_{S_j} \right)^i & \coloneqq \psi^6 \sqrt{\tilde{\gamma}/\hat{\gamma}} \left(S_j \hat{v}^i + \delta^i{}_j \alpha p \right), \\
    \left(f_\tau \right)^i & \coloneqq \psi^6 \sqrt{\tilde{\gamma}/\hat{\gamma}} \left(\tau \hat{v}^i + \alpha p v^i \right),
\end{align}
where $\hat{v}^i \coloneqq \alpha v^i -\beta^i$.\\
Finally, the source terms $\mathbf{s}$ are
\begin{align}
    s_D &= 0, \\
    s_{S_i} &= \alpha \psi^6 \sqrt{\tilde{\gamma}/\hat{\gamma}} 
    \left[ - T^{00} \alpha\partial_i \alpha + T^0{}_k \hat{\nabla}_i \beta^k
    + \frac{1}{2} \left( T^{00} \beta^j \beta^k + 2 T^{0j} \beta^k + T^{jk} \right) \hat{\nabla}_i \gamma_{jk} \right], \\
    s_\tau &= \alpha \psi^6 \sqrt{\tilde{\gamma}/\hat{\gamma}}
    \left[ T^{00} \left(K_{ij}\beta^i \beta^j - \beta^k \partial_k \alpha \right)
    + 2 T^{0j} \left( 2 K_{jk} \beta^k - \partial_j \alpha \right) 
    + T^{ij} K_{ij} \right].
\end{align}
