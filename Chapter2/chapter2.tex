%!TEX root = ../thesis.tex
%*******************************************************************************
%*********************************** Second Chapter *****************************
%*******************************************************************************

\chapter{Formulations of Einstein Field Equations}  %Title of the Second Chapter

\ifpdf
    \graphicspath{{Chapter2/Figs/PDF/}{Chapter2/Figs/}}
\else
    \graphicspath{{Chapter2/Figs/}}
\fi


%********************************** %First Section  **************************************
\section{Introduction} %Section - 2.1
\label{section2.1}

Due to the complexity and nonlinearity of Einstein field equations, it is extremely difficult to obtain analytical solution even for the simplest dynamical evolution systems.
Therefore, the accurate discription of the such systems can only be derived through numerical simulation.
For this, we need to reformulate the Einstein equations as an initial-value problem or Cauchy problem.
In this chapter, we will introduction the Arnowitt-Deser-Misner (ADM) formulation, which is the foundation of the 3+1 numerical relativity.
In particular, we will focus on the constrained scheme for the Einstein equations.

%********************************** %Second Section  *************************************
\section{The 3+1 decomposition of spacetime} %Section - 2.2 
\label{section2.2}
\subsection{Foliation of spacetime} \label{section2.2.1}

In the 3+1 decomposition, the spacetime manifold $\mathcal{M}$ is foliated into a set of non-intersecting spacelike hypersurfaces $\Sigma_t$ parameterized by the coordinate time $t$ \cite{misner1973gravitation}.
We denote a future-directed timelike unit four-vector $n^\mu$ normal to the hypersurface $\Sigma_t$ (i.e. $n_{\mu} \propto \nabla_\mu t$).
The induced spacetime metric $\gamma_{\mu\nu}$ on each hypersurfuce can then be defined as
\begin{align}\label{eq:2.2.spatial}
    \gamma_{\mu\nu} \coloneqq g_{\mu\nu} + n_{\mu} n_{\nu}.
\end{align}
Thus, we can construct spatial projection tensor $\gamma^{\mu}{}_{\nu}$ and time projection tensor $N^{\mu}{}_{\nu}$ as
\begin{align}
    \gamma^{\mu}{}_{\nu} &\coloneqq \delta^{\mu}{}_{\nu} + n^{\mu} n_{\nu}, & N^{\mu}{}_{\nu} &\coloneqq - n^{\mu} n_{\nu},
\end{align}
which decompose any generic four-vector $U^\mu$ into spatial part $\gamma^{\mu}{}_{\nu}U^{\nu}$ and timelike part $N^{\mu}{}_{\nu}U^{\nu}$.
Therefore, we can decompose the timelike vector field
\begin{align}\label{eq:2.2.1.normal_decompose}
t^\mu = \alpha n^{\mu} + \beta^{\mu}
\end{align}
into two components as
\begin{align}
    \alpha &\coloneqq - t^\mu n_\mu, & \beta^{\mu} &\coloneqq t^{\nu} \gamma^{\mu}{}_{\nu},
\end{align}
where the lapse function $\alpha$ measures the physical proper time ($\alpha\Delta t$) between two neighboring spatial hypersurface $\Sigma_t$ and $\Sigma_{t+\Delta t}$
, and the shift vector $\beta^i$ measures the changes of spatial coordinates on $\Sigma_{t+\Delta t}$.\\
Here, we summarise several useful relations.
The timelike normal vector $n^\mu$ and its corresponding one-form $n_\mu$ can be expressed as
\begin{align} \label{eq:2.2.1.normal}
    n^\mu &= \frac{1}{\alpha}\left(1, \beta^i \right), & n_\mu &= \left(\alpha, \vec{0}, \right).
\end{align}
The generic line element in 3+1 decomposition is given by
\begin{align}
    ds^2 = - \left( \alpha^2 - \beta^i \beta_i \right) dt^2 + \beta_i dx^i dt + \gamma_{ij} dx^i dx^j
\end{align}
The covariant and contravariant components of the metric can be written as
\begin{align}\label{eq:2.2.1.g}
    g_{\mu\nu} &= 
    \begin{pmatrix}
        - \alpha^2 + \beta^i \beta_i & \beta_j \\
        \beta_i & \gamma_{ij}
    \end{pmatrix}, &
    g^{\mu\nu} &= 
    \begin{pmatrix}
        - \frac{1}{\alpha^2} & \beta^j \\
        \beta^i & \gamma^{ij}
    \end{pmatrix}.
\end{align}
From equation(\ref{eq:2.2.1.g}), we can conclude that
\begin{align}
    \sqrt{-g} = \alpha \sqrt{\gamma},
\end{align}
where $g \coloneqq \det{\left(g_{\mu\nu}\right)}$ and $\gamma \coloneqq \det{\left(\gamma_{ij}\right)}$.

\subsection{Derivative operator} \label{section2.2.2}
With the 3+1 decomposition, we can now construct the 3-dimensional covariant derivative $D_\alpha$ associated with $\gamma_{\mu\nu}$ by projecting the 4-dimensional covariant derivative $\nabla_\alpha$ onto $\Sigma_t$, which is given by
\begin{align}\label{eq:2.2.2.spatialD}
    D_{\alpha} T^{\mu_1 \mu_2 \dots}{}_{\nu_1 \nu_2 \dots} = \gamma_{\alpha}{}^{\beta}\gamma_{\rho_1}{}^{\mu_1} \gamma_{\rho_2}{}^{\mu_2} \dots \gamma_{\nu_1}{}^{\sigma_1} \gamma_{\nu_2}{}^{\sigma_2} \dots \nabla_{\beta} T^{\rho_1 \rho_2 \dots}{}_{\sigma_1 \sigma_2 \dots},
\end{align}
for arbitrary tensor $T^{\mu_1 \mu_2 \dots}{}_{\nu_1 \nu_2 \dots}$ on spatial hypersurface $\Sigma_t$.
Using equation(\ref{eq:2.2.2.spatialD}), it can be shown that the corvariant derivative of $\gamma_{\mu\nu}$ vanishes
\begin{align}
\begin{split}
    D_{\alpha} \gamma_{\mu\nu} &= \gamma_{\alpha}{}^{\beta} \gamma_{\rho}{}^{\mu} \gamma_{\nu}{}^{\sigma} \nabla_{\beta} \left(g_{\rho\sigma} + n_\rho n_\sigma \right) \\
    &= \gamma_{\alpha}{}^{\beta} \gamma_{\rho}{}^{\mu} \gamma_{\nu}{}^{\sigma} \left( n_\rho \nabla_\beta n_\sigma + n_\sigma \nabla_\beta n_\rho \right) = 0
\end{split}
\end{align}
The components of 3-dimensional connection coefficients $\Gamma^{\alpha}{}_{\mu\nu}$ in coordinate basis can be expressed as
\begin{align}\label{eq:3_connection}
    \Gamma^{\alpha}{}_{\mu\nu} &= \frac{1}{2} \gamma^{\alpha\beta} \left( \partial_{\nu}\gamma_{\beta\mu} + \partial_{\mu}\gamma_{\beta\nu} - \partial_{\beta}\gamma_{\mu\nu} \right).
\end{align}
Here, the upper left index ${}^{(4)}$ marks the 4-dimensional tensors while the unmarked one represents purely spatial 3-dimensional tensors.
Similarly, the 3-dimensional Riemann tensor $R^{\alpha}{}_{\beta\mu\nu}$ associated with $\gamma_{\mu\nu}$ is defined by requiring that
\begin{align}
    2 D_{[\nu} D_{\mu]} W_\beta &= W_\alpha R^\alpha{}_{\beta\mu\nu}, & R^{\alpha}{}_{\beta\mu\nu} n_\alpha &= 0,
\end{align}
which can be explicitly expressed in coordinate basis as
\begin{align}
    R^{\alpha}{}_{\beta\mu\nu} &= \partial_{\mu} \Gamma^{\alpha}{}_{\beta\nu} - \partial_{\nu} \Gamma^{\alpha}{}_{\beta\mu} + \Gamma^{\alpha}{}_{\mu\rho}\Gamma^{\rho}{}_{\beta\nu} - \Gamma^{\alpha}{}_{\nu\rho}\Gamma^{\rho}{}_{\beta\mu}.
\end{align}
The 3-dimensional Ricci tensor $R_{\mu\nu}$ and Ricci scalar $R$ are defined in a similar manner as their 4-dimensional counterparts
\begin{align}
     R_{\mu\nu} &\coloneqq  R^\alpha{}_{\mu\alpha\nu}, & R &\coloneqq  R^{\mu}{}_{\mu}.
\end{align}
Since $R^{\alpha}{}_{\beta\mu\nu}$ is purely spatial and can be computed by the spatial derivatives of the spatial metric alone,
it only contains about information about the curvature intrinsic to the hypersurface $\Sigma_t$,
but cannot contain all the information of ${}^{(4)} R^{\alpha}{}_{\beta\mu\nu}$ which includes time derivative of the 4-dimensional metric.
The missing information can be found in a purely spatial symmetric tensor called the extrinsic curvature $K_{\mu\nu}$.

\subsection{Extrinsic curvature} \label{section2.2.3}
The extrinsic curvature $K_{\mu\nu}$ is related to the time derivative of the spatial metric $\gamma_{\mu\nu}$.
Therefore, the spatial metric and extrinsic curvature $\left(\gamma_{\mu\nu}, K_{\mu\nu} \right)$ are equivalent to the positions and velocities in classical mechanics,
which describe the instantaneous state of the gravitational field.
It can be obtained by projecting of the gradient of the normal vector $\gamma_{\mu}{}^{\lambda}\gamma_{\nu}{}^{\rho} \nabla_{\lambda} n_{\rho}$ into the hypersurface $\Sigma_t$,
and then taking the negative expression of the symmetric part
\begin{align}\label{eq:2.2.3.K1}
\begin{split}
   K_{\mu\nu} \coloneqq& - \gamma_{\mu}{}^{\lambda}\gamma_{\nu}{}^{\rho} \nabla_{\lambda} n_{\rho} \\
   =& - \gamma_{\mu}{}^{\lambda} \left( \delta_{\nu}{}^{\rho} + n_{\nu} n^{\rho} \right) \nabla_{\lambda} n_{\rho} \\
   =& - \gamma_{\mu}{}^{\lambda} \nabla_{\lambda} n_{\nu},
\end{split}
\end{align}
where the identity $n^{\rho}\nabla_{\lambda} n_{\rho}=0$ is used. \\
We can also define an spatial acceleration $a_\nu$
\begin{align}
    a_\nu \coloneqq n^\mu \nabla_\mu n_\nu,
\end{align}
satisfying the identities
\begin{align}
    a_\nu = D_\nu \ln{\alpha},
\end{align}
to rewrite equation(\ref{eq:2.2.3.K1}) as
\begin{align}
    K_{\mu\nu} = - \nabla_\mu n_\nu - n_\mu a_\nu
\end{align}
Finally, we can write the extrinsic curvature $K_{\mu\nu}$ as the Lie derivative of the spatial metric along the local normal $n^\mu$
\begin{align}
    K_{\mu\nu} = - \frac{1}{2} \mathcal{L}_{n} \gamma_{\mu\nu}.
\end{align}
Using equation(\ref{eq:2.2.1.normal_decompose}), we can express the Lie derivative $\mathcal{L}_n$ as
\begin{align}\label{eq:normal_Lie}
    \mathcal{L}_n = \frac{1}{\alpha} \left( \mathcal{L}_t - \mathcal{L}_\beta \right),
\end{align}
and thus obtain the evolution equation for the spatial metric
\begin{align}\label{eq:g_evol}
    \mathcal{L}_t \gamma_{\mu\nu} = - 2\alpha K_{\mu\nu} + \mathcal{L}_\beta \gamma_{\mu\nu}.
\end{align}

\subsection{The Gauss, Codazzi and Ricci equations}
\label{section2.2.4}

To express the Einstein field equations in term of the spatial variables $(\gamma_{\mu\nu}, K_{\mu\nu})$ we defined previous,
we first have to relate 3-dimensional Riemann tensor $R^\alpha{}_{\beta\mu\nu}$ on $\Sigma_t$ to the 4-dimensional Riemann tensor ${}^{(4)}R^\alpha{}_{\beta\mu\nu}$ on $\mathcal{M}$,
The relation between $R^\alpha{}_{\beta\mu\nu}$ and the full spatial projection of ${}^{(4)}R^\alpha{}_{\beta\mu\nu}$ is given by the Gauss' equation
\begin{align}\label{eq:Gauss}
    R_{\alpha\beta\mu\nu} + K_{\alpha\mu}K_{\beta\nu} - K_{\alpha\nu} K_{\beta\mu} = \gamma_{\alpha}{}^{\rho} \gamma_{\beta}{}^{\sigma} \gamma_{\mu}{}^{\lambda} \gamma_{\nu}{}^{\delta} {}^{(4)}R_{\rho\sigma\lambda\delta},
\end{align}
while the projection of ${}^{(4)}R^\alpha{}_{\beta\mu\nu}$ with one index projected in the normal direction is given by the Codazzi equation
\begin{align}\label{eq:Codazzi}
    D_{\nu} K_{\mu\alpha} - D_{\mu} K_{\nu\alpha} = \gamma_{\mu}{}^{\rho} \gamma_{\nu}{}^{\sigma} \gamma_{\alpha}{}^{\lambda} n^{\delta} {}^{(4)}R_{\rho\sigma\lambda\delta}.
\end{align}
Finally, by projecting two indices of ${}^{(4)}R_{\rho\sigma\lambda\delta}$ in the normal direction,
we can relate it to the time derivative of $K_{\mu\nu}$
\begin{align}\label{eq:Ricci}
    \mathcal{L}_n K_{\mu\nu} = n^{\alpha} n^{\beta} \gamma_{\mu}{}^{\lambda} \gamma_{\nu}{}^{\delta} {}^{(4)}R_{\alpha\deta\beta\lambda} - \frac{1}{\alpha} D_{\mu} D_\nu \alpha - K_{\nu}{}^{\lambda}K_{\mu\lambda},
\end{align}
which is called the Ricci equation.

\subsection{Constraint and evolution equations} %Section - 2.2.5
\label{section2.2.5}

Using the Gauss, Codazzi and Ricci equations,
the Einstein fields equations can be decomposed into a set of evolution equations and a set of constraint equations of $(\gamma_{\mu\nu}, K_{\mu\nu})$.
To begin with, we define the following matter quantities
\begin{align}
    S_{\mu\nu} \coloneqq& \gamma^{\alpha}{}_{\mu} \gamma^{\beta}{}_{\nu} T_{\alpha\beta}, \\
    S_\mu \coloneqq& - \gamma^{\alpha}{}_{\mu} n^{\beta} T_{\alpha\beta}, \\
    S \coloneqq& S^\mu{}_\mu, \\
    E \coloneqq& n^\alpha n^\beta T_{\alpha\beta},
\end{align}
which decompose the stress-energy tensor as
\begin{align}
    T_{\mu\nu} = E n_\mu n_\nu + S_\mu n_\nu + S_\nu n_\mu + S_{\mu\nu}.
\end{align}\\
By contracting the $\alpha,\mu$ indices in equation(\ref{eq:Gauss}), we can obtain
\begin{align}\label{eq:Gauss_1st}
    R_{\mu\nu} = \gamma_{\mu}{}^{\alpha}\gamma_{\nu}{}^{\beta} \left( {}^{(4)} R_{\alpha\beta} + n^\rho n^\sigma {}^{(4)}R_{\alpha\rho\beta\sigma} \right) + K_{\mu\lambda} K_\nu{}^{\lambda} - K_{\mu\nu} K,
\end{align}
where $K\coloneqq K^\mu{}_{\mu}$ is the trace of the extrinsic curvature.
Further contracting the $\mu, \nu$ indices in equation(\ref{eq:Gauss_1st}), the contracted Gauss' equation becomes
\begin{align}\label{eq:Gaussr_2nd}
    2 n^\mu n^\nu G_{\mu\nu} = R + K^2 - K_{\mu\nu} K^{\mu\nu},
\end{align}
Using the Einstein equation(\ref{eq:Einstein_eq}), we can obtain the Hamiltonian constraint
\begin{align}
    R + K^2 - K_{\mu\nu} K^{\mu\nu} = 16\pi E.
\end{align}
Similarly, by contracting $\alpha, \nu$ indices in equation(\ref{eq:Codazzi}), the Codazzi equation yields
\begin{align}
    D_{\nu} K_{\mu}{}^{\nu} - D_\mu K = - 8\pi \gamma^{\alpha}{}_{\mu} n^{\beta} T_{\alpha\beta},
\end{align}
and thus obtain the momentum constrain equation
\begin{align}
    D_{\nu} K_{\mu}{}^{\nu} - D_\mu K = 8 \pi S_{\mu}.
\end{align}\\
The Ricci equation (\ref{eq:Ricci}) can be rewritten using equation(\ref{eq:Gauss_1st}) to
\begin{align}
    \mathcal{L}_n K_{\mu\nu} = R_{\mu\nu} - \gamma_{\mu}{}^{\rho} \gamma_\nu{}^{\sigma} {}^{(4)} R_{\rho\sigma} - 2 K_{\mu\lambda} K_\nu{}^{\lambda} + K K_{\mu\nu} - \frac{1}{\alpha} D_\mu D_\nu \alpha.
\end{align}
Using the Einstein equations (\ref{eq:Einstein_eq})
\begin{align}
\begin{split}
    \gamma_{\mu}{}^{\rho} \gamma_\nu{}^{\sigma} {}^{(4)} R_{\rho\sigma} &= 8\pi \gamma_{\mu}{}^{\rho} \gamma_\nu{}^{\sigma} \left( T_{\rho\sigma} - \frac{1}{2}g_{\rho\sigma} T^\mu{}_\mu \right)\\
    &= 8\pi \left[S_{\mu\nu} - \frac{1}{2} \gamma_{\mu\nu} \left(S-E\right) \right]
\end{split}
\end{align}
and equation(\ref{eq:normal_Lie}), we can finally obtain the evolution equation for $K_{\mu\nu}$ as
\begin{align}\label{eq:K_evol}
    \mathcal{L}_t K_{\mu\nu} = - D_\mu D_\nu \alpha + \alpha \left(R_{\mu\nu} - 2K_{\mu\lambda} K_\nu{}^{\lambda} + K K_{\mu\nu} \right) 
    - 8 \pi \alpha \left[ S_{\mu\nu} - \frac{1}{2}\gamma_{\mu\nu} \left(S-E\right) \right] + \mathcal{L}_\beta K_{\mu\nu}.
\end{align}

\subsection{The Arnowitt, Deser and Misner equations} %Section - 2.2.6
\label{section2.2.6}

The Lie derivative in the evolution equations (\ref{eq:g_evol}) and (\ref{eq:K_evol}) can be expressed in terms of coordinate basis as
\begin{align}
    \mathcal{L}_t K_{\mu\nu} &= \partial_t K_{\mu\nu} \\
    \mathcal{L}_\beta K_{\mu\nu} &= \beta^\lambda D_\lambda K_{\mu\nu} + K_{\mu\lambda} D_\nu \beta^\lambda + K_{\lambda\nu} D_\mu \beta^\lambda
\end{align}\\
As the result, the Einstein field equaitons (\ref{eq:Einstein_eq}) in the standard 3+1 decomposition can be decomposed into a set of constraint equations and evolution equation of $(\gamma_{ij}, K_{ij})$ in terms of coordinate basis,
which are referred to as the Arnowitt, Deser and Misner (ADM) equations \cite{amowitt1962dynamics,york1979kinematics}
\begin{flalign}
    & R + K^2 - K_{ij}K^{ij} = 16\pi E, && \text{(Hamiltonian constraint)} \label{eq:ADM_H_const}\\
    & D_j \left(K^{ij} - \gamma^{ij} K \right) = 8\pi S^i, && \text{(momentum constraint)} \label{eq:ADM_S_const}\\
    & \partial_t \gamma_{ij} = - 2 \alpha K_{ij} + D_i \beta_j + D_j \beta_i, && \text{(spatial metric evolution)} \label{eq:ADM_g_evol}\\
    &
    \begin{aligned}
    \partial_t K_{ij} =& - D_i D_j \alpha + \alpha \left(R_{ij} - 2 K_{ik} K_j{}^k + K K_{ij} \right) \\
    &- 8\pi \alpha \left[S_{ij} - \frac{1}{2}\gamma_{ij} \left(S-E \right) \right] \\
    &+ \beta^k D_k K_{ij} + K_{ik} D_j \beta^k + K_{kj} D_i \beta^k.
    \end{aligned} && \text{(extrinsic curvature evolution)} \label{eq:ADM_K_evol}
\end{flalign}

%********************************** % Third Section  *************************************
\section{Conformal Decomposition} %Section - 2.3
\label{section2.3}

The conformal decompostion factors out a scalar component from a spatial metric.
It was first developped for initial data problems in general relativity \cite{lichnerowicz1944integration,york1971gravitational,york1972role,york1973conformally},
and then used in reformulating evolution equations in the 3+1 formulation.
In this section, we will discuss the conformal decomposition of spatial metric and extrinsic curvature in numerical relativity.


\subsection{Conformal transformation of the spatial metric}
\label{section2.3.1}
We consider the conformal transformation of the spatial metric $\gamma_{ij}$ as
\begin{align}\label{eq:conformal_metric}
    \tilde{\gamma}_{ij} = \psi^{-4} \gamma_{ij},
\end{align}
where $\tilde{\gamma}_{ij}$ is the \textit{conformal metric} and $\psi$ is a positive scaling factor satisfying
\begin{align}
    \psi &\coloneqq \det{\left( \frac{\gamma}{f} \right)}, & \gamma &\coloneqq \det{\left( \gamma_{ij} \right)}, & f &\coloneqq \det{\left( f_{ij} \right)}
\end{align}
for a time independent flat metric $f_{ij}$ (i.e. $\det{\left( \tilde{\gamma}_{ij} \right)} = f$ by construction).\\
Thus, the \textit{inverse conformal metric} is given by
\begin{align}\label{eq:conformal_metric_inv}
    \tilde{\gamma}^{ij} \coloneqq \psi^4 \gamma^{ij}.
\end{align}
Substituting the conformal transformation (\ref{eq:conformal_metric}) into equation(\ref{eq:3_connection}),
we can obtain the transformation law for 3-dimensional connection coefficient
\begin{align}
    \Gamma^{i}{}_{jk} = \tilde{\Gamma}^{i}{}_{jk} + 2 \left( \delta^i{}_j \tilde{D}_k \ln \psi + \delta^i{}_k \tilde{D}_j \ln \psi
    - \tilde{\gamma}_{jk} \tilde{\gamma}^{il} \tilde{D}_l \ln \psi \right).
\end{align}
From now on, we denote all objects associated with the conformal metric $\tilde{\gamma}^{ij}$ with a tilde symbol.
Similarly, the transformation for Ricci tensor and scalar curvature are given by
\begin{align}
\begin{split}
    R_{ij} =& \tilde{R}_{ij} - 2 \left( \tilde{D}_i \tilde{D}_j \ln \psi + \tilde{\gamma}_{ij} \tilde{\gamma}^{lm} \tilde{D}_l \tilde{D}_m \ln \psi \right)\\
    &+ 4 \left( \left(\tilde{D}_i \ln \psi \right) \left( \tilde{D}_j \ln \psi \right) - \tilde{\gamma}_{ij} \tilde{\gamma}^{lm} \tilde{D}_l \left(\tilde{D}_l \ln \psi \right) \left( \tilde{D}_m \ln \psi \right) \right)
\end{split} \label{eq:conformal_ricci_tensor}\\
    R =& \psi^{-4} \tilde{R} - 8 \psi^{-5} \tilde{D}^2 \psi, \label{eq:conformal_ricci_scalar}
\end{align}
where $\tilde{D}^2 = \tilde{\gamma}^{ij} \tilde{D}_i \tilde{D}_j$ denotes the Laplace operator associated with $\tilde{\gamma}_{ij}$.
Therefore, using equation(\ref{eq:conformal_ricci_scalar}), the Hamiltonian constraint (\ref{eq:ADM_H_const}) becomes
\begin{align}
    8 \tilde{D}^2 \psi - \psi \tilde{R} - \psi^5 K^2 + \psi^2 K_{ij} K^{ij} = - 16 \pi \psi^5 E.
\end{align}

\subsection{Conformal transformation of the extrinsic curvature}
\label{section2.3.2}

\subsubsection{Traceless decomposition}
Before we perform the conformal transformation to the extrinsic curvature $K_{ij}$, it is convenient to split $K_{ij}$ into the trace part 
\begin{align}
    K \coloneqq \gamma^{ij} K_{ij},
\end{align}
and its traceless part
\begin{align}
    A_{ij} &\coloneqq K_{ij} - \frac{1}{3} \gamma_{ij} K, & \operatorname{tr}_{\gamma} A_{ij} &= \gamma^{ij} A_{ij} = 0.
\end{align}
Therefore, we can obtain the tracelss decomposition of the extrinsic curvature
\begin{align}
    K_{ij} &= A_{ij} + \frac{1}{3} \gamma_{ij} K, & K^{ij} &= A^{ij} + \frac{1}{3} \gamma^{ij} K.
\end{align}
The evolution equations of the spatial metric (\ref{eq:ADM_g_evol}) in conformal deposition formulation can hence be written as
\begin{flalign}
    \partial_t \psi &= \beta^i \tilde{D}_i \psi - \frac{1}{6}\psi \left(\alpha K - \tilde{D}_i \beta^i \right), &&\text{(conformal factor evolution)} \label{eq:psi_evol} \\
    \partial_t \tilde{\gamma}_{ij} &= - 2\alpha \psi^{-4} A_{ij} + \tilde{\gamma}_{jk} \tilde{D}_i \beta^k + \tilde{\gamma}_{ik} \tilde{D}_j \beta^k - \frac{1}{3} \tilde{\gamma}_{ij} \tilde{D}_k \beta^k, &&\text{(conformal metric evolution)} \label{eq:con_g_evol_1}
\end{flalign}
and the constraint equations become
\begin{flalign}
    &\tilde{D}^2 \psi - \frac{1}{8} \psi \tilde{R} + \left( \frac{1}{8} A_{ij} A^{ij} - \frac{1}{12}K^2 + 2 \pi E \right) \psi^5 = 0, 
    &&\text{(Hamiltonian constraint)} \label{eq:H_const_s} \\
    &\tilde{D}_i \left(\psi^{10} A^{ij} \right) - \frac{2}{3}\psi^6 \tilde{D}^i K = 8\pi \psi^{10} S^i. 
    &&\text{(momentum constraint)} \label{eq:S_const_s}
\end{flalign}

\subsubsection{Conformal transformation of the traceless part}
We consider the transformation
\begin{align}
    A^{ij} \coloneqq \psi^a \tilde{A}^{ij},
\end{align}
for some undetermined exponent $\alpha$.
Here, we discuss two natural choices of $a$: $a = -4$ and $a = -10$.

\subparagraph{“Time-evolution” scaling: $a=-4$.}
This choice of scaling was considered by Nakamura in 1994 \cite{nakamura19943d},
It comes naturally from the evolution equation of conformal metric (\ref{eq:con_g_evol_1}),
where the $\psi^{-4}A_{ij}$ term suggests the conformal transformation of $A_{ij}$ to have the same scaling factor as the conformal spatial metric (\ref{eq:conformal_metric_inv})
\begin{align}
    \tilde{A}^{ij} \coloneqq \psi^4 A^{ij},
\end{align}
where the indices of $\tilde{A}^{ij}$ and $\tilde{A}_{ij}$ are lowered and raised by the conformal metric $\tilde{\gamma}_{ij}$ and $\tilde{\gamma}^{ij}$ respectively
(i.e. $\tilde{A}_{ij} = \tilde{\gamma}_{il}\tilde{\gamma}_{jm}\tilde{A}^{lm} = \psi^{-4} A_{ij}$).
The evolution equations of conformal spatial metric therefore become
\begin{align}
    \partial_t \tilde{\gamma}_{ij} &= - 2\alpha \tilde{A}_{ij} + \tilde{\gamma}_{jk} \tilde{D}_i \beta^k + \tilde{\gamma}_{ik} \tilde{D}_j \beta^k - \frac{1}{3} \tilde{\gamma}_{ij} \tilde{D}_k \beta^k.
\end{align}
The Hamiltonian constraint and momentum constraint in this scaling is rewritten as
\begin{align}
    &\tilde{D}^2 \psi = \frac{1}{8} \psi \tilde{R} - \left( 2 \pi E - \frac{1}{12}K^2 + \frac{1}{8} \tilde{A}_{ij} \tilde{A}^{ij} \right) \psi^5, 
    \label{eq:H_const_s4}\\
    &\tilde{D}_i \left(\psi^6 \tilde{A}^{ij} \right) - \frac{2}{3}\psi^6 \tilde{D}^i K = 8\pi \psi^{10} S^i.\label{eq:S_const_s4}
\end{align}

\subparagraph{“Momentum-constraint” scaling: $a=-10$.}
Another possible choice of scaling factor $a=-10$ originates from the momentum constraint equation (\ref{eq:S_const_s}),
which was first suggested by Lichnerowicz in 1944 \cite{lichnerowicz1944integration}.
we define
\begin{align}
    \hat{A}^{ij} \coloneqq \psi^{10} A^{ij},
\end{align}
and thus
\begin{align}
    \hat{A}_{ij} = \psi^{2} A_{ij}.
\end{align}
Here, we use hat symbol to separate the "momentum-constraint" scaling from the tilde symbol in "time-evolution" scaling.
These two scaling are related by
\begin{align}
    \hat{A}^{ij}&= \psi^6\tilde{A}^{ij}, &\hat{A}_{ij} &= \psi^6 \tilde{A}_{ij}, &\hat{A}_{ij} \hat{A}^{ij} &= \psi^{12} \tilde{A}_{ij} \tilde{A}^{ij}
\end{align}
Therefore, the constraint equations can be written as
\begin{align}
    &\tilde{D}^2 \psi = \frac{1}{8} \psi \tilde{R} - \left( 2\pi E - \frac{1}{12}K^2 \right) \psi^5 - \frac{1}{8} \hat{A}_{ij} \hat{A}^{ij} \psi^{-7}, 
    \label{eq:H_const_s10}\\
    &\tilde{D}_i \hat{A}^{ij} - \frac{2}{3}\psi^6 \tilde{D}^i K = 8\pi \psi^{10} S^i.\label{eq:S_const_s10}
\end{align}
Equation(\ref{eq:S_const_s10}) is known as \textit{Lichnerowicz equation}.
Although equation (\ref{eq:S_const_s4}) and equation (\ref{eq:S_const_s10}) are equivalent,
they have different mathematical properties if they are treated as a partial differential equation for $\psi$.
According to the maximal principle,
the local uniqueness of the solutions depends on the sign of the exponent of $\psi$ in the quadratic extrinsic curvature $A^2$ term 
\cite{cordero2009improved,smarr1979sources,taylor1991partial,evans1997partial,protter2012maximum}.
Equation (\ref{eq:S_const_s4}) suffers from the mathematical nonuniqueness problems due to the positive exponent $(+5)$,
while the negative exponent $(-7)$ in equation (\ref{eq:S_const_s10}) guarantees the local uniquess of the solutions.

\subsection{Conformal transverse-traceless decomposition}
Using the "momentum-constraint” scaling mentioned previously,
we can further decompose the symmetric, traceless tensor $\hat{A}^{ij}$ as
\begin{align}
    \hat{A}^{ij} = \hat{A}^{ij}_{TT} + \hat{A}^{ij}_{L},
\end{align}
where $\hat{A}^{ij}_{TT}$ is the transverse-traceless part which is divergenceless
\begin{align}
    \tilde{D}_j \hat{A}^{ij}_{L} = 0,
\end{align}
and $\hat{A}^{ij}_{L}$ is the longitudinal part satisfying
\begin{align}
    \hat{A}^{ij}_{L} = \tilde{D}^i X^j + \tilde{D}^j X^i - \frac{2}{3}\tilde{\gamma}^{ij} \tilde{D}_k X^k \equiv \left(\tilde{L} W \right)^{ij}.
\end{align}
The vector $X^i$ here is the vector potential 
and $\tilde{L}$ is the \textit{longitudinal operator} or \textit{conformal Killing operator} associated with $\tilde{\gamma}$
which gives a symmetric, traceless tensor.
The divergence of $\hat{A}^{ij}$ becomes
\begin{align}
    \tilde{D}_j \hat{A}^{ij} = \tilde{D}^2 X^i + \frac{1}{3} \tilde{D}^i \left( \tilde{D}_j X^j \right).
\end{align}
Thus, the momentum constraint in the conformal transverse-traceless (CTT) decomposition yields
\begin{align}
    \tilde{D}^2 X^i + \frac{1}{3} \tilde{D}^i \left( \tilde{D}_j X^j \right) - \frac{2}{3}\psi^6 \tilde{D}^i K = 8 \pi \psi^{10} S^i,
\end{align}
with the Hamiltonian constraint same as equation (\ref{eq:H_const_s10}).

%********************************** % Forth Section  *************************************
\section{Gauge Condition}  %Section - 2.4
\label{section2.4}

\subsection{Maximal Slicing} 
\label{section2.4.1}

\subsection{Generalized Dirac Gauge}
\label{section2.4.2}

%********************************** % Fifth Section  *************************************
\section{Constrained scheme for the Einstein equations}  %Section - 2.5
\label{section2.5}

\subsection{The Conformal Flatness Condition}  %Section - 2.5.1
\label{section2.5.1}

\subsection{The Fully Constrained Formulation}  %Section - 2.5.2
\label{section2.5.2}

