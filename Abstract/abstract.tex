% ************************** Thesis Abstract *****************************
% Use `abstract' as an option in the document class to print only the titlepage and the abstract.
\begin{abstract}
Many astrophysical scenarios involve neutron stars and black holes such as core-collapse supernovae,
binary neutron star and neutron star-black hole mergers,
which are the most important events in gravitational wave physics and multimessenger astrophysics.
To numerically model these systems accurately within a reasonable time and affordable computational resources,
a multi-scale, multi-dimensional,
fully parallelized, support different geometries general relativistic
(magneto-)hydrodynamics code is desired.
We developed our new open-source, parallelized, block-grid adaptive,
multi-dimensional general relativistic electro-magneto-hydrodynamics code \texttt{Gmunu}
(General-relativistic multigrid numerical solver)
in curvilinear geometries in dynamical spacetimes.\\
\texttt{Gmunu} code current adopts the conformal flatness condition approximation 
by solving the elliptic metric equations by multigrid approach.
Although the \texttt{Gmunu} code pass with flying colors in benchmarking tests and demanding magnetohydrodynamics problems,
the conformal flatness condition approximation prohibits the emission of gravitational wave in the simulation.
Numerical simulations with full general-relativistic treatment are able to 
simulation gravitational wave signals emitted from various astrophysics events such as compact binary coalescence and core-collapse supernova.
Therefore, it is necessary to extend our code to adopt the fully constrained formulation,
which is a full general relativistic formulation naturally generalized from the conformal flatness condition approximation.
In contrast to the fully hyperbolic formulations of general relativity such as BSSN and CCZ4,
the fully constrained formulation maximizes the number of elliptic equations in each time steps
by choosing maximal slicing condition and generalized Dirac gauge,
leaving only two degrees of freedom in the hyperbolic sector which corresponds to the gravitational wave in far field regime.
With the usage of multigrid solver,
the elliptic metric equations can be solved multi-dimensionally in Cartesian, cylindrical or spherical geometries.\\
We present the methodology, implementation details and the performance of
the fully constrained formulation in \texttt{Gmunu}.
\end{abstract}
