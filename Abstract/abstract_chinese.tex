% ************************** Thesis Abstract *****************************
% Use `abstract' as an option in the document class to print only the titlepage and the abstract.
\begin{abstract_chinese}
\begin{CJK*}{UTF8}{bsmi}
{\CJKfamily{bkai}
許多天體物理場景都涉及中子星和黑洞,
例如核心坍縮超新星、雙中子星和中子星-黑洞合併,
它們是引力波物理學和多信使天體物理學中最重要的事件。
為了在合理的時間和負擔得起的計算資源內準確地對這些系統進行數值模擬,
需要多尺度、多維、完全並行、支持不同幾何形狀的通用相對論(磁)流體動力學代碼。
我們開發了新的開源、並行化、塊網格自適應、多維廣義相對論電磁流體動力學代碼 \underline{豬緲妞}(廣義相對論多重網格數值求解器),
用於動態時空曲線幾何。\\
\underline{豬緲妞} 碼流採用共形平坦度條件近似,
通過多重網格方法求解橢圓度量方程。
儘管 \underline{豬緲妞} 代碼在基準測試和苛刻的磁流體動力學問題中表現出色,
但保形平坦度條件近似阻止了模擬中引力波的發射。
具有完全廣義相對論處理的數值模擬能夠模擬從各種天體物理事件發出的引力波信號,
例如緻密二元聚結和核心坍縮超新星。
因此,
有必要擴展我們的代碼以採用完全約束公式,
這是從保角平坦度條件近似自然推廣的完全廣義相對論公式。
與廣義相對論的全雙曲公式相比,
全約束公式通過選擇最大切片條件和廣義狄拉克規範,
在每個時間步中最大化橢圓方程的數量,
雙曲扇區只留下兩個自由度這對應於遠場區域的引力波。
使用多重網格求解器,
橢圓度量方程可以在笛卡爾、圓柱或球面幾何中進行多維求解。\\
我們介紹了 \underline{豬緲妞} 中完全約束公式的方法、實現細節和性能。}
\end{CJK*}
\end{abstract_chinese}
