%!TEX root = ../thesis.tex
%*******************************************************************************
%*********************************** Fifth Chapter *****************************
%*******************************************************************************

\chapter{Concluding Remarks}  %Title of the Fifth Chapter

\ifpdf
    \graphicspath{{Chapter5/Figs/PDF/}{Chapter5/Figs/}}
\else
    \graphicspath{{Chapter5/Figs/}}
\fi
The main theme of this work is to extend the \texttt{Gmunu} code to adopt the
fully constrained formulation (FCF) scheme,
a full general relativistic formulation naturally generalized from the conformally flat condition (CFC) approximation
which is current used in the code.
So far only two dimensional cylindrical coordinate is implemented for the FCF scheme.
In this thesis, we present the methodology and implementation of \texttt{Gmunu} in detail.\\
We have tested \texttt{Gmunu} with several benchmarking tests.
In the Teukolsky wave test, we have shown the stability and convergence of evolving the hyperbolic equations in FCF.
We also demonstrated that the elliptic divergence cleaning using multigrid method is able to maintain the gauge condition.
For the hydrodynamics,
we have performed simulations of the evolution of both non-rotating and rotating neutron stars.
We are able to extract the gravitational wave signature directly from the metric components.

%********************************** %First Section  **************************************
\section*{Future Plan}
Here, we list some possible improvement of \texttt{Gmunu} planned for the future.

\paragraph{Support different geometries}
Currently, the FCF scheme is only implemented in 2D cylindrical coordinate
which limit our study to axisymmetric profiles.
We will extend the code to support 2D spherical
as well as 3D spherical/Cartesian coordinates in the future
so that we can broaden our study to non-axisymmetric systems
such as binary systems and supernovae.

\paragraph{Simulate black holes}
In \texttt{Gmunu}, we have no treatment to handle the black holes singularity nor the apparent horizon.
It has been shown that the excision scheme can be used to deal with black holes for constrained evolution scheme in spherically symmetric spacetime \cite{cordero2014excision}.
Therefore, we would like to develop a horizon finder in \texttt{Gmunu} and implement the excision scheme in the future.

\paragraph{Improve the FCF scheme}
We used finite difference for solving Einstein equations and finite volume for solving the hydrodynamics.
Although they are equivalent in the case of second order accuracy,
this becomes a problem if we extend the code to high-order numerical methods in the future.
It has been suggested that the hyperbolic sector in FCF can be rewritten in conservative form \cite{cordero2008mathematical}.
Therefore, we would like to rewrite the hyperbolic equations in FCF to suit the finite volume scheme in the future.
